\documentclass[14pt]{extarticle}
\usepackage{multicol}
\usepackage[utf8]{vietnam}
\usepackage[left = 2cm, right = 2cm]{geometry}
\usepackage{amsmath}
\usepackage{amsfonts}
\usepackage{amssymb}
\usepackage{mathtools}
\usepackage{changepage}
\title{Software Engineering Code - ACM Ethics}
\author{Nguyễn Minh Hiển}


\usepackage{titlesec}
% Redefine section format
\titleformat{\section}
  {\normalfont\Large\bfseries}{Principle \thesection}{1em}{}

% Redefine subsection format
\titleformat{\subsection}
{\normalfont\normalsize\bfseries}{\thesubsection}{1em}{}

\begin{document}
\newpage
\maketitle
\tableofcontents
\newpage

\section{PUBLIC}
\subsection{Accept full responsibility for their own work}
Ví dụ: Khi một kĩ sư phần mềm được nhiệm vụ phát triển một tính
năng mới cho một phần mềm và trong quá trình làm việc, người này phát
hiện ra một lỗi lớn có ảnh hưởng nghiêm trọng đến người dùng. \\
"Chịu trách nhiệm hoàn toàn cho công việc của mình" có nghĩa là người này
phải có trách nhiệm thông báo về lỗi đó cho người quản lý và 
cùng với đồng nghiệp tìm cách giải quyết vấn đề trước khi triển khai 
tính năng mới đó. Nếu không, người này sẽ phải chịu trách nhiệm 
về hậu quả của lỗi đó.

\setcounter{subsection}{2}
\subsection{Approve software only if they have a well-founded 
belief that it is safe, meets specifications, passes appropriate 
tests, and does not diminish quality of life, diminish privacy or 
harm the environment. The ultimate effect of the work should be to 
the public good.}
Ví dụ: Trước khi phát hành một phần mềm liên quan đến thông tin 
cá nhân của người dùng, nhóm phát triển phần mềm cần phải tiến 
hành kiểm tra kỹ lưỡng để đảm bảo phần mềm đáp ứng được các tiêu chuẩn
an toàn, không rõ rỉ dữ liệu ảnh hưởng đến quyền riêng tư của người 
dùng, hạn chế việc các hacker tấn công ăn cắp thông tin cá nhân. Họ
cũng cần đảm bảo phần mềm mang lại lợi ích chung cho cộng đồng, tránh
gây hậu quả xấu đến phần lớn người sử dụng.

\section{CLIENT AND EMPLOYER}
\subsection{Provide service in their areas of competence, being 
honest and forthright about any limitations of their experience 
and education.}
Ví dụ: Trong quá trình làm việc, nếu kỹ sư phần mềm được giao nhiệm vụ
phát triển một tính năng mà người này không có kinh nghiệm hoặc kiến thức
về lĩnh vực đó, người này cần phải thẳng thắn trao đổi với người 
quản lý và đồng nghiệp để tìm ra giải pháp tốt nhất. Thay vì cố gắng
làm việc ở lĩnh vực mà bản thân không đủ hiểu biết, người đó có thể
được phân công nhiệm vụ khác phù hợp với năng lực của mình hoặc
học hỏi và cải thiện kiến thức của mình từ đồng nghiệp.

\subsection{Not knowingly use software that is obtained or retained 
either illegally or unethically}
Ví dụ: Khi sử dụng một phần mềm, chúng ta phải đọc kỹ các điều khoản
sử dụng và đảm bảo rằng phần mềm đó được cài đặt và sử dụng hợp pháp.
Không sử dụng các phần mềm bản quyền không hợp pháp (như việc crack
hay lợi dụng các lỗ hổng để không trả phí sử dụng phần mềm).
Không sử dụng phần mềm có chứa mã độc hại, virus, trojan, worm... 
gây phát tán các mã độc này sang các máy tính khác.

\section{PRODUCT}
\setcounter{subsection}{6}
\subsection{Strive to fully understand the specifications for 
software on which they work.}
Ví dụ: Trước khi bắt đầu phát triển một phần mềm, kỹ sư phần mềm cần
dành thời gian nghiên cứu các đặc tả của phần mềm, hiểu rõ yêu cầu
về chức năng, giao diện người dùng, hiệu suất và bảo mật. Họ cần phải
thảo luận một cách chi tiết với bộ phận kỹ thuật và bộ phận chăm
sóc khách hàng để đảm bảo hiểu đúng về yêu cầu của khách hàng đối với
phần mềm.

\section{JUDGMENT}
\setcounter{subsection}{2}
\subsection{Maintain professional objectivity with respect to any 
software or related documents they are asked to evaluate.}
Ví dụ: Khi một nhóm kiểm thử phần mềm tiến hành việc kiểm tra, họ 
phải đảm bảo rằng mọi chức năng của phần mềm đều phải được kiểm tra
và đánh giá một cách công bằng, khách quan. Dù có mối quan hệ quen biết
với nhà phát triển hay khách hàng thì vẫn luôn phải giữ vững sự chuyên
nghiệp và không để yếu tố cá nhân nào ảnh hưởng đến quá trình đánh giá.
Như vậy, kết quả đánh giá sẽ đáng tin cậy và chính xác.

\section{MANAGEMENT}
\setcounter{subsection}{3}
\subsection{Assign work only after taking into account appropriate 
contributions of education and experience tempered with a desire to 
further that education and experience.}
Ví dụ: Khi phân công công việc cho các nhân viên, người quản lý cần
phải xem xét kỹ lưỡng về năng lực, kinh nghiệm và kiến thức của từng
nhân viên để đảm bảo công việc được thực hiện một cách hiệu quả. Tuy
nhiên, cũng không nên chỉ dựa vào bằng cấp hay kinh nghiệm trước đó, 
mà nên dựa vào cả mong muốn được học hỏi và phát triển để giúp đỡ,
cho họ cơ hội thử thách và phát triển kỹ năng mới, có thể để cho người
này hỗ trợ các đồng nghiệp đã có năng lực, kinh nghiệm trước đó.

\section{PROFESSION}
\setcounter{subsection}{9}
\subsection{Avoid associations with businesses and organizations 
which are in conflict with this code.}
Ví dụ: Một kỹ sư phần mềm được mời tham gia vào một dự án phần mềm với 
một công ty có liên quan đến việc thu thập và sử dụng dữ liệu cá nhân 
một cách không minh bạch (như việc hack các tài khoản mạng xã hội) 
hoặc kinh doanh một sản phẩm bất hợp pháp, trái đạo đức (như văn hóa
phẩm đồi trụy, cờ bạc). Dù dự án có thể mang lại nhiều lợi ích kinh tế
nhưng hoạt động của công ty đó có thể xâm phạm quyền riêng tư người dùng
hay trái với pháp luật, đạo đức. Vì vậy, thay vì tham gia vào dự án đó, 
ta cần phải từ chối và tìm cách phản ánh với cơ quan chức năng để ngăn
chặn hoạt động xâm phạm quyền lợi người dùng của công ty.

\section{COLLEAGUES}
\setcounter{subsection}{2}
\subsection{Credit fully the work of others and refrain from taking undue credit}
Ví dụ: Khi vừa hoàn thành một sản phẩm và công bố, công lao của tất cả
mọi người tham gia vào dự án phần mềm đó cần được công nhận đầy đủ.
Tất cả mọi người đóng góp đều đáng được khen ngợi và công nhận
từ đồng nghiệp và quản lý. Không lợi dụng vị trí, quyền lực hay quan
hệ để chiếm đoạt công lao của người khác hay đưa ra công lao không
thuộc về mình. Điều này thể hiện sự tôn trọng đối với đóng góp của người khác
và giúp họ có dộng lực để tiếp tục làm việc.

\section{SELF}
\setcounter{subsection}{1}
\subsection{Improve their ability to create safe, reliable, and 
useful quality software at reasonable cost and within a
reasonable time.}
Ví dụ: Để sản xuất phần mềm chất lượng cao, an toàn và đáng tin cậy, 
kỹ sư phần mềm cần không ngừng nâng cao khả năng của mình trong 
thời đại công nghệ liên tục phát triển, cập nhật. Ta dành thời gian 
học hỏi và thực hành các kỹ thuật mới, cải thiện quy trình phát triển 
và áp dụng các phương pháp kiểm tra và đánh giá. Ta cũng cần tìm cách 
tối ưu hóa quá trình làm việc và sản phẩm của mình để giảm thiểu 
chi phí và thời gian, mà vẫn đảm bảo rằng sản phẩm cuối cùng đạt được 
chất lượng mong muốn.
\end{document}
