\documentclass[14pt]{extarticle}
\usepackage[utf8]{vietnam}
\usepackage[left = 2cm, right = 2cm]{geometry}
\usepackage{amsmath}
\usepackage{amsfonts}
\usepackage{amssymb}
\usepackage{mathtools}
\usepackage{changepage}

\begin{document}
\newpage
\tableofcontents
\newpage

\setcounter{section}{5}
\section{Quality model for external and internal quality}

\subsection{Functionality}

\setcounter{subsubsection}{1}
\subsubsection{Accuracy}

Accuracy (tính chính xác) của phần mềm là khả năng của phần mềm cung cấp
các kết quả hoặc ảnh hưởng đúng hay chấp nhận được ở một mức độ
chính xác cần thiết.

Để đo tính chính xác của phần mềm, em sẽ chia phần mềm ra các
chức năng cơ bản. Với từng chức năng, đưa ra các bộ test mà
đã biết trước kết quả (có thể sinh ra bằng tay hoặc bằng các công
cụ khác như Excel, ...) và sau đó, chạy phần mềm và đếm số
test mà phần mềm có thể chạy đúng.

\textbf{Ví dụ:} Về tính chính xác của một phần mềm diệt virus (AV).\\
Em sẽ chia chức năng chính của phần mềm này thành 2 việc:
không nhận diện được mã độc và nhận diện nhầm một phần mềm hợp lệ.
Về mã độc, chia làm 2 loại: mã độc mới, chưa có trong cơ sở dữ liệu
của AV (để kiểm tra khả năng phân tích hành vi của AV) và mã độc
phổ biến.\\
Các loại mã độc, phần mềm sẽ được thu thập từ các cơ sở dữ liệu đáng tin cậy
trên Internet để xem AV có thể nhận diện được bao nhiêu mã độc cũng như
nhận diện nhầm bao nhiều phần mềm hợp lệ.

\subsubsection{Interoperability}
Interoperability (tính tương tác) của phần mềm là khả năng của phần mềm
tương tác với một hay nhiều hệ thống cụ thể khác nhau.\\
Để đo tính tương tác của phần mềm, em sẽ xem xét hiệu suất và chất lượng
của phần mềm trong các tình huống tương tác khác nhau. Cụ thể hơn, em đặt
các chức năng khác nhau của phần mềm vào các tình huống tương tác khác nhau
nhiều lần để thống kê được xác suất thành công, thất bại, chất lượng, hiệu suất
khi tương tác với các hệ thống đó.

\textbf{Ví dụ:} Về tính tương tác của một phần mềm mạng xã hội (SN).\\
Em cần thống kê được xác suất thành công, thất bại, chất lượng, hiệu suất
sử dụng SN trên các hệ điều hành khác nhau
(Windows, Linux, MacOS, Android, iOS, ...), các trình duyệt khác nhau
(Chrome, Firefox, Safari, Opera, ...), các thiết bị khác nhau
(máy tính, điện thoại, máy tính bảng, ...).\\
Tiếp theo đó, em cũng cần xem xét về khả năng trao đổi dữ liệu của SN đó
với hệ thống bảo mật, hệ thống quảng cáo, hệ thống thanh toán, ... khác.


\subsection{Reliability}
\subsubsection{Maturity}
Maturity (tính trưởng thành) là khả năng tránh lỗi của phần mềm
sau những lần gặp lỗi của phần mềm.\\
Điều này có thể được đo bằng số lần xảy ra lỗi và số lần lỗi được khắc phục.
Cụ thể hơn ở ví dụ bên dưới.

\textbf{Ví dụ:} Về tính "trưởng thành" của phần mềm diệt virus Kaspersky.\\
Phần mềm đã xuất hiện trên thị trường từ năm 2006 với
số lượng bản vá lỗi vào khoảng 100 bản được công bố trên trang
web chính thức của Kaspersky. Thêm vào đó, cơ sở dữ liệu mã độc luôn
được cập nhật liên tục nhằm giảm thiểu khả năng lỗi của phần mềm.

\subsubsection{Fault tolerance}
Fault tolerance (tính chịu lỗi) của phần mềm là khả năng của phần mềm
duy trì hiệu suất ở một mức độ cụ thể trong trường hớp phần mềm gặp lỗi
hoặc vi phạm giao diện cụ thể của nó.\\
Em đề xuất việc đo khoảng thời gian giữa 2 lần phần mềm gặp lỗi,
và khoảng thời gian để phần mềm khắc phục lỗi.
Cụ thể hơn: Em sẽ chọn một khoảng thời gian ngẫu nhiên rồi đo tổng thời gian lỗi,
số lần lỗi. Sau đó, sử dụng các công cụ Xác suất thống kê, tính toán
khoảng thời gian trung bình giữa 2 lần lỗi
và khoảng thời gian trung bình để khắc phục lỗi của phần mềm đó.\\
khoảng thời gian trung bình giữa 2 lần lỗi càng lớn
và khoảng thời gian trung bình để khắc phục lỗi càng nhỏ chứng tỏ tính chịu lỗi
của phần mềm càng cao.

\subsection{Usability}
\subsubsection{Understandability}
Understandability (tính dễ hiểu) là khả năng của phần mềm cho phép người dùng
hiểu được liệu phần mềm có phù hợp và có thể sử dụng trong nhiệm vụ và
điều kiện cụ thể.\\
Một phần mềm có dễ hiểu hay không sẽ phụ thuộc vào độ phức tạp của phần mềm.
Cụ thể, độ phức tạp của phần mềm có thể sử dụng các chỉ số phần mềm Halstead
(Halstead’s Software metrics):
program length, vocabulary size, program volume, program difficulty,
program effort, time to implement.
Các chỉ số này càng cao thì phần mềm càng phức tạp và khó hiểu.

\subsubsection{Learnability}
Learnability (tính dễ học) là khả năng của phần mềm cho phép người dùng
học cách ứng dụng phần mềm.\\
Để đo tính dễ học của phần mềm, em sẽ đo thời gian mà người dùng cần
để học các chức năng từ cơ bản đến nâng cao của phần mềm.

\textbf{Ví dụ:} Về tính dễ học của một phần mềm soạn thảo (ví dụ Word).\\
Em sẽ thực hiện khảo sát trên 3 nhóm người dùng chưa biết gì về Word:
trẻ học tiểu học, thanh niên, và trung niên,
phân theo giới tính nam, nữ, và phân theo hiểu biết về máy tính.
Em sẽ đo thời gian mà từng nhóm người dùng cần để học cách sử dụng
các chức năng cơ bản và nâng cao của Word.
Từ cách đo này có thể đánh giá được tính dễ học của Word đối với
nhóm chức năng cơ bản hay nâng cao của Word, đối với từng nhóm người dùng.





\subsection{Efficiency}
\subsubsection{Time behavior}
Time behavior là khả năng của phần mềm cung cấp
phản hồi, thời gian xử lý và tốc độ thông lượng phù hợp khi thực hiện các
chức năng trong điều kiện đã nêu.\\
Rõ ràng, cách đo time behavior sẽ phụ thuộc vào chức năng cụ thể của phần mềm
và cũng được nêu rõ trong định nghĩa.
Em cần đo thời gian phản hồi, thời gian xử lý và tốc độ thông lượng của phần mềm
trong các khoảng thời gian ngẫu nhiên và sau đó, sử dụng các công cụ Xác suất
thống kê để đánh giá time behavior của phần mềm.

\subsubsection{Resource utilization}
Resource utilization (tính tận dụng tài nguyên) là khả năng của phần mềm
sử dụng lượng và loại tài nguyên phù hợp khi thực hiện các chức năng trong
điều kiện đã nêu.\\
Để đo tính tận dụng tài nguyên của phần mềm, em cần đo lượng tài nguyên
cần thiết để thực hiện các chức năng cụ thể của phần mềm (như lượng RAM,
CPU, GPU, ổ cứng, ...) cũng như xem phần mềm đó có tận dụng đúng tài nguyên
(RAM hay VRAM, CPU hay GPU, ...) không. Những tài nguyên nào được sử dụng,
cùng lượng tài nguyên mà một phần mềm (hay service) đang sử dụng có thể đo
đơn giản nhất trên Windows là sử dụng Task Manager để đo.


\subsection{Maintainability}
\setcounter{subsubsection}{2}
\subsubsection{Stability}
Stability (tính ổn định) là khả năng của phần mềm tránh ảnh hưởng không mong
muốn từ việc sửa đổi phần mềm.\\


\subsection{Portability}


\end{document}
